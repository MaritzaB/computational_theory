\documentclass{article}
\usepackage[utf8]{inputenc}
\usepackage[spanish,es-tabla,es-nodecimaldot]{babel}
\usepackage{enumitem}
\usepackage{graphicx}
\graphicspath{ {reports/figures/} }
\usepackage{textcomp}
\usepackage{color}
\usepackage{upquote,listings}

\input{custom/sql_stylo}

\title{Normalización de base de datos del Directorio Estadístico de Unidades Económicas (DENUE).}
\author{
    Ana Maritza Bello Yañez
}

\date{\today}

\begin{document}
\maketitle


\begin{abstract}
    \subsection*{Objetivos}
    \begin{enumerate}
        \item Crear catálogos a partir de atributos que no tengan dependencia a
        otros atributos.
        \item Sustituir las llaves primarias de los catálogos creados en la base
        de datos original, de tal manera que las tablas estén relacionadas por
        llave primaria y foránea.
        \item Verificar que el espacio ocupado en disco por la base de datos
        disminuyó con la creación de catálogos.
    \end{enumerate}
\end{abstract}

\section{Introducción}

\subsection*{Normalización de bases de datos}

La normalización de una base de datos consiste en re-diseñarla en estructuras
más pequeñas. Esto se hace mediante la transformación de las vistas de usuario
complejas y de la base de datos a un juego de tablas de datos más pequeñas y
estables (catálogos).

Cuando una base de datos crece, al no tener un buen diseño, se crean varios
problemas; entre ellos la redundancia de datos, que es cuando repetimos la misma
información en muchos registros. Esta redundancia en los datos se traduce en
espacio en disco y con esto se crea también un problema de mantenimiento, ya que
si queremos actualizar un dato, es posible que tengamos que hacerlo en más de un
lugar, con lo que crece la posibilidad de cometer errores.

La finalidad de la normalización es mejorar el desempeño de la base de datos, es
decir, mejorar el diseño y eliminar datos que sean redundantes. Así, cuando
tengamos que actualizar un dato en una base de datos normalizada, solo lo
tendremos que cambiar en una tabla y evitaremos errores a la hora de
actualizarlo.

Cuando normalizamos una base de datos reducimos su espacio en disco ya que se
crean tablas más pequeñas sin datos repetidos. Las tablas generadas (catálogos)
estarán relacionadas entre sí a partir de una llave primaria o
\textit{primary\_key}, que será una columna que llevará la palabra
\textit{id} de identificador en el nombre. 

Los datos de la llave primaria generalmente son del tipo \textit{integer} ya que
los datos de este tipo ocupan un espacio de almacenamiento de 4 bytes, mientras
que un dato tipo \textit{string} de 15 caractéres usará hasta 19 bytes.

\subsection*{Datos del DENUE}

Para realizar esta práctica utilizamos los datos del DENUE 2022. Estos datos
contienen la información de 5,528,698 negocios en el país e incorpora las
actualizaciones de establecimientos, en especial, a partir del Estudio sobre la
Demografía de los Negocios 2021.

Los datos que publica el directorio sobre las unidades económicas son los
siguientes:

\begin{enumerate}
    \item Identificación
    \begin{itemize}
        \item Nombre de la unidad económica
        \item Denominación o razón social (personas morales)
        \item Estrato de personal ocupado
        \item Código y título o nombre de la clase de actividad económica
        \item Tipo de la unidad económica
    \end{itemize}

    \item Ubicación
    \begin{itemize}
        \item Domicilio postal o geográfico
        \item Tipo y nombre de vialidad
        \item Número exterior
        \item Edificio, piso o nivel
        \item Número interior
        \item Tipo y nombre del asentamiento humano
        \item Corredor industrial, centro comercial o mercado público
        \item Número de local
        \item Código Postal
        \item Área geoestadística Estatal (AGEE)
        \item Área geoestadística Municipal (AGEM)
        \item Localidad geoestadística
        \item Manzana
        \item Coordenadas de Latitud
        \item Coordenadas de Longitud
    \end{itemize}

    \item Contacto
    \begin{itemize}
        \item Número de teléfono
        \item Correo electrónico
        \item Sitio en internet
    \end{itemize}
    \item Fecha de incorporación al DENUE
\end{enumerate}

\section{Desarrollo}
Como primer paso renombramos las columnas de tal manera que los nombres quedaran
bien especificados y evitar ambigüedades. Ya que estamos trabajando solo con
datos de la ciudad de México, también se eliminaron las columnas de
\texttt{clave\_entidad} y \texttt{entidad}, ya que estas contenían los datos
\texttt{09} y \texttt{Ciudad de México} respectivamente, repetidos en todos los
registros.

\lstinputlisting[language=SQL,   
framesep=10pt,
framextopmargin=10pt]
{../src/rename_columns.sql}

\subsection{Creación de catálogos}
Para reducir el espacio ocupado en disco de la base de datos del DENUE, creamos
6 catálogos de datos. Se escogieron atributos que fueran independientes de
otros, los catálogos creados fueron los siguientes:

\begin{enumerate}
    \item Catálogo de municipios.
    Los municipios o alcaldías de la Ciudad de México son 16 y ya que son un
    atributo independiente de los datos del DENUE (no cambia), optamos por hacer
    un catálogo de este.
    \lstinputlisting[language=SQL,   
        framesep=10pt,
        framextopmargin=10pt]
        {../src/municipios.sql}

    \item Catálogo de actividades económicas.
    Los tipos de actividades económicas ya están definidas por el DENUE, así que
    también es conveniente hacer un catálogo de estas.
    \lstinputlisting[language=SQL,   
    framesep=10pt,
    framextopmargin=10pt]
    {../src/actividades.sql}

    \item Catálogo de vialidades.
    Los tipos de vialidades que existen son 24. La conveniencia de crear un
    catálogo de estas es que tenemos 4 campos en los cuales podemos sustituir el
    valor por la llave foránea de este catálogo.
    \lstinputlisting[language=SQL,   
    framesep=10pt,
    framextopmargin=10pt]
    {../src/vialidades.sql}

    \item Catálogo de tipos de asentamientos.
    Los tipos de asentamiento humano ya están definidos, por lo que no cambiarán
    y es independiente de otro atributo. Los tipos de asentamientos son 43.
    \lstinputlisting[language=SQL,   
    framesep=10pt,
    framextopmargin=10pt]
    {../src/asentamientos.sql}

    \item Catálogo de centros comerciales.
    Se refiere al tipo de plaza, centro comercial, etc. donde se encuentra la
    unidad económica.
    \lstinputlisting[language=SQL,   
    framesep=10pt,
    framextopmargin=10pt]
    {../src/centros_comerciales.sql}

    \item Catálogo de contactos de establecimientos.
    Este catálogo se creo con la finalidad de crear una tabla que contenga los
    datos de contacto del negocio, estos son:

    \begin{itemize}
        \item Teléfono
        \item Correo electrónico
        \item Página web
    \end{itemize}

    Dado que son atributos cuyo valor puede ser nulo o no, el hecho de ponerlos
    en una tabla aparte nos permite eliminar de la base de datos principal
    aquellos registros donde los tres campos están vacíos.

    \lstinputlisting[language=SQL,   
    framesep=10pt,
    framextopmargin=10pt]
    {../src/datos_de_contacto.sql}

\end{enumerate}


\subsection{Reemplazo de columnas por llave foránea}


\section{Resultados y discusión}

\begin{table}[h!]
    \centering
\begin{tabular}{c|c|c}
    Catálogo sustituido & Tamaño del catálogo & Tamaño de la base de datos\\
    \hline
    Base de datos original & No aplica & 236 MB \\
    Municipios & 24 KB & 222 MB \\
    Actividades & 152 KB & 195 MB \\
    Vialidades & 24 KB & \\
    Asentamientos & 24 KB & \\
    Centros comerciales & 24 KB & \\
    Datos de contacto & 24 MB &
\end{tabular}
\end{table}

\section{Conclusiones}

\section*{Apéndice}


%\begin{figure}[t]
%    \includegraphics[width=13cm]{figures/bd_original.png}
%    \centering
%\end{figure}

\end{document}