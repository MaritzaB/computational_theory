\documentclass{article}
\usepackage[utf8]{inputenc}
%\usepackage[spanish,es-tabla,es-nodecimaldot]{babel}
\usepackage{listings}
\usepackage{enumitem}
\usepackage{setspace}
\usepackage{parskip}
\usepackage{tikz}
\usetikzlibrary{automata, positioning, arrows}
\tikzset{ ->, % makes the edges directed
>=stealth', % makes the arrow heads bold
node distance=2cm, % specifies the minimum distance between two nodes. Change if
necessary.
every state/.style={thick, fill=gray!10},
% sets the properties for each ’state’ node
initial text=$ $ }

%\input{custom/specifications}

\title{Exercises from Introduction to the Theory of Computation. Chapter 7. 
\\ Michael Sipser.}
\author{Ana Maritza Bello Yañez}
\date{\today}

\begin{document}
\maketitle

\textbf{7.1} Answer each part TRUE or FALSE.
\begin{itemize}
    \item a. $2n = O(n)$    Verdadero
    \item b. $n^2 = O(n)$   Falso
    \item c. $n^2 = O(n log^2 n)$   Falso
    \item d. $n log n = O(n^2)$   Verdadero
    \item e. $3^n = 2^{O(n)}$  Verdadero
    \item f. $2^{2^n} = O(2^{2^n})$ Verdadero
\end{itemize}

\textbf{7.2} Answer each part TRUE or FALSE.
    \begin{itemize}
        \item a. $n = o(2n)$    Falso
        \item b. $2n = o(n^2)$  Verdadero
        \item c. $2^n = o(3^n)$ Verdadero
        \item d. $1 = o(n)$ Verdadero
        \item e. $n = o(log n)$ Falso
        \item f. $1 = o(1/n)$   Falso
    \end{itemize}

\pagebreak

\textbf{7.3} Which of the following pairs of numbers are relatively prime? Show
the calculations that led to your conclusions.

\begin{itemize}
    \item a. 1247 and 10505
    \item b. 7289 and 8029
\end{itemize}

Utilizaremos el siguiente algoritmo para determinar si dos números son primos.

\lstinputlisting{../src/euclidex.py}

\texttt{>> El maximo comun divisor entre 10505 y 147 es (-67, 4788, 1)}. Por lo
tanto son primos relativos.

\texttt{>> El maximo comun divisor entre 7289 y 8029 (-76, 69, 37)}. No son primos
relativos. \\


\textbf{7.4} Fill out the table described in the polynomial time algorithm for
context-free language recognition from Theorem 7.16 for string $w = baba$ and
\textit{CFG G:}

\begin{itemize}
    \item $S \rightarrow RT$
    \item $R \rightarrow TR|a$
    \item $T \rightarrow TR|b$
\end{itemize}




\textbf{7.5} Is the following formula satisfiable?
$ (x \lor y) \land (x \lor \neg{y}) \land (\neg{x} \lor y) \land (\neg{x} \lor
\neg{y}) $

Podemos evaluar todas las posibles combinaciones de valores para las variables x
e y y determinar si alguna hace que la fórmula sea verdadera. Hay cuatro
combinaciones posibles:

    x = verdadero, y = verdadero:
    
    En este caso, 
    $ (x \lor y)$ es verdadera ($V or V = V$),

    $(x \lor \neg{y})$ es verdadera ($V or F = V$),

    $(\neg{x} \lor y)$ es verdadera ($F or V = V$),

    $(\neg{x} \lor \neg{y})$ es verdadera ($F or F = F$).

    Por lo tanto la fórmula completa es verdadera.

    x = verdadero, y = falso:
    
    En este caso, 
    $ (x \lor y)$ es verdadera ($V or F = V$),

    $(x \lor \neg{y})$ es verdadera ($V or V = V$),

    $(\neg{x} \lor y)$ es verdadera ($F or F = V$),

    $(\neg{x} \lor \neg{y})$ es falso ($F or V = V$).

    Por lo tanto la fórmula completa es falsa.

    x = falso, y = verdadero:
    
    En este caso, 
    $ (x \lor y)$ es verdadera ($F or V = V$),

    $(x \lor \neg{y})$ es falso ($F or F = F$),

    $(\neg{x} \lor y)$ es verdadera ($V or V = V$),

    $(\neg{x} \lor \neg{y})$ es verdadera ($V or F = V$).

    Por lo tanto la fórmula completa es falsa.

    x = falso, y = falso:

    En este caso, 
    $ (x \lor y)$ es falso ($F or F = F$),

    $(x \lor \neg{y})$ es verdadero ($V or F = V$),

    $(\neg{x} \lor y)$ es verdadero ($F or V = V$),

    $(\neg{x} \lor \neg{y})$ es verdadero ($F or F = V$).

    Por lo tanto la fórmula completa es falsa.
    

En resumen, la fórmula no es satisfacible porque no hay al menos una asignación
de valores $x,y$ que la haga verdadera.

\textbf{7.6} Show that P is closed under union, concatenation, and complement.



\textbf{7.7} Show that NP is closed under union and concatenation.

\end{document}